\documentclass{article}
\usepackage[margin=.75in]{geometry}
\begin{document}

\newcommand{\ts}{\textsuperscript}

\centerline{\huge \textbf{Aiden Cullo}}\newline
\centerline{\small 446 Circle Avenue, Kingston, NY 12401 \textbar \, 845.943.8311 \textbar \,  \textit{culloaiden3@gmail.com} \textbar \,  \textit{http://www.aidencullo.me}}\newline

\textbf{EDUCATION}
\vspace{10mm}
\rule{\linewidth}{0.4pt}
Binghamton University, Thomas J. Watson School of Engineering 	                                                     Binghamton, NY
•	Bachelor of Science in Computer Science, Applied Physics                                                                                       Expected May 2019
•	Cumulative GPA: 3.73/4.0; Dean’s list; Phi Eta Sigma Honor Society

TECHNICAL SKILLS
•	Coursework: Data Structures | Programming with Objects | Machine Organization | Computer Systems | Architecture and Programming | Integral Calculus | Multivariable Calculus | Modern Physics | Differential Equations
•	Computer Science: Java, C, C++, Python, Linux, Excel, Javascipt, SQL, git, bash, LaTeX, Vim, OpenMP

RELEVANT EXPERIENCE
Binghamton Department of Physics, Applied Physics, and Astronomy	       	                                       Binghamton, NY
Research Assistant			   				                                                                   August 2016—Present
•	Write software for evolutionary machine learning algorithm to find most stable atomic configuration
•	Debug and parallelize C code using multithreading with OpenMP
•	Collaborate with fellow undergraduate and graduate students on lab experiments
Cornell Robotics Personal Assistants Laboratory				          		                                    Ithaca, NY
Research Intern			     	              						                  May 2016—August 2016
•	Developed algorithm in C++ and ROS for robot to visually identify properties of an object (shape, texture, etc.)
•	Wrote data marshaling software in Python to communicate instructions/information between distributed components
•	Coordinated with other team members to modify software in order to smoothly integrate into larger project

LEADERSHIP
Hack BU	        								            		            Binghamton, NY
Internal Hackathon Coordinator							                                      August 2015—Present
•	Help organize and run Binghamton’s annual hackathon with 300+ participants and volunteers
•	Communicate with other hackathon organizers, sponsors, and school administrators
•	Mentor and teach student hackers at weekly meetings and club sponsored events
Binghamton Robotics Club						                 		             Binghamton, NY
President									              	                                      August 2016—Present
•	Coordinate teams to participate in inter-club design competitions and the Spaceport America Cup
•	Organize demonstrations, group projects and collaborative events with other clubs
•	Educate graduate/undergraduate members on concepts related to physics, computer science, and robotics

PROJECTS
Artemis II											            Binghamton, NY
Hybrid-Fuel Rocket								                                      August 2016—Present
•	Designed shape and structure of 6-foot-tall frame and nosecone of rocket
•	Managed a group of 5-6 engineers and computer scientists in creating rocket components
•	Built prototype for modular rocket engine powered by a potassium nitrate and sucrose hybrid fuel
DNA Sequencer							                                                                      New York, NY
Python compiler that calculates melting temperature		                                                                    December 2016—January 2017
•	Developed algorithm in Python that calculates Gibb’s free energy contribution of each base pair
•	Using oligonucleotide molarity and salt content, calculated the temperature of 50\% dissociation of given DNA strand
•	Corresponded with computation biology department at NYU to perform DNA experiments on double mismatch sequences
Pippin										                                       Binghamton, NY
Java CPU simulator					                                           	      	                January 2016—May 2016
•	Graphical User Interface that imitates a CPU with memory slots, an accumulator, a program counter, etc.
•	Assembled pippin assembly (.pasm) source code into executable files and runs .exe file in GUI
•	Simulates a CPU by changing memory slots as each machine code commands are executed
Mazerunner									                      	                Rosendale, NY
Autonomous VEX Robot         								               February 2015—May 2015
•	Built robot and attached servo motors, ultrasonic sensors, bump sensors, microcontroller and rechargeable battery
•	Calibrated motors/sensors through extensive tests and reassembling of robot
•	Wrote software in easyC and Python that utilized sensor data to avoid obstacles and navigate through maze

\end{document}
